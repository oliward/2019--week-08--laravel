So far, most of the code you've written has been ``procedural'': start at the top of a file, run through it, maybe call a few functions as you go, and then finish at the end. This is fine for simple programs or when we're just working inside an existing system (e.g. WordPress), but it doesn't really scale to larger applications.
\\

The problem comes because we need to manage \textbf{state}: keeping track of all the values in our code. For a large app you could easily have thousands of values that need storing. Naming and keeping track of all these variables would become a nightmare if they were all in the same global scope.
\\

\textbf{Object-Oriented Programming}\footnote{``Oriented'' not ``Orientated''.} (OOP) is one way to make this easier. The key idea behind OOP is \textbf{encapsulation}: we keep functions and variables that are related to each other in one place (an object) and then use visibility to limit which bits of code can access and change them.
\\

An object is effectively a black box: objects can send \textbf{messages} to each other (by calling methods), but they need not have any knowledge of the inner workings of other objects.


\pagebreak


\begin{infobox}{The Unusual History of PHP}
    PHP has a long and complicated history. The first version of PHP wasn't even a programming language, it was just simple templating language that allowed you to re-use the same HTML code in multiple files.

    \quoteinline{I don't know how to stop it, there was never any intent to write a programming language \textellipsis{} I have absolutely no idea how to write a programming language, I just kept adding the next logical step on the way.}{Rasmus Lerdorf, Creator of PHP}

    Over the years PHP morphed into a simple programming language and then into a modern object-oriented programming language. However, it wasn't really until 2009, with the release of PHP 5.3, that PHP could truly be considered a fully object-oriented language.
    \\

    Because of this gradual change, older PHP frameworks and systems (such as WordPress) were originally written using non-OO code, which is why they still contain a large amount of procedural code.
    \\

    PHP gets a lot of flack for not being a very good programming language and a few years ago that was perhaps a valid criticism. But in recent years, particularly with the release of PHP 7, it's just not true anymore. It certainly still has some issues, but nothing that a few libraries can't get around.

    \quoteinline{There are only two kinds of languages: the ones people complain about and the ones nobody uses}{Bjarne Stroustrup, Creator of C++}
\end{infobox}

\pagebreak


\section{Encapsulation}

Say that our app includes some code to send an email. If we were using procedural code we would probably have a function called \texttt{sendMail} that we can pass various values to:

\phpinputminted{02/figures/01/01-mail}

But we might want to be able to customise more than just the to, from, and message parts of the email. Which means we'd either need to have a lot of optional arguments (which becomes unwieldy quickly) or rely on global variables:

\phpinputminted{02/figures/01/02-global-mail}

But this is truly horrible: we have no way of preventing other parts of our code from changing these values and we would start having to use long variable names to avoid ambiguity in bigger apps.
\\

So, we want to store the variables and the functionality together in one place and in such a way that values can't be accidentally changed. This is where objects come in:

\phpinputminted{02/figures/01/03-mail-class}

Now if we need to add additional fields, we can just add a property and setter method


\section{(Almost) Pure OO}

Many object oriented languages \textit{only} use objects. For example in Java everything lives inside a class and you specify which class your app should create first when you run it.
\\

Because of PHP's history we always need a little bit of procedural code to get our objects up and running. This is often called the \textbf{bootstrap} file.

\phpinputminted{02/figures/01/04-bootstrap}

Once we've created our first object the idea of object-oriented programming is that we use objects from that point onwards.


\section{Object Types \& Type Declarations}

We looked at ``types'' with JavaScript: numbers, strings, booleans, arrays, \&c. PHP has a similar set of built-in types. In OOP we generally talk of objects of having the type of their class: e.g. a \texttt{Person} object instance is of the type \texttt{Person}.
\\

PHP supports \textbf{type declarations}.\footnote{In previous versions of PHP these were called type hints} These let us say that the arguments passed to a function or method must be of a certain type. For example, say we had a \texttt{MailingList} class with a \texttt{sendWith()} method. It would be useful to say that we can only pass \texttt{Mail} objects into this method\footnote{This is an example of \textbf{dependency injection}.}:

\phpinputminted{02/figures/01/05-MailingList}

Before accepting the \texttt{\$mailer} parameter, we add the type declaration/hint of \texttt{Mail}. Now, if the user of that class tries to pass in something that isn't an instance of \texttt{Mail}, PHP will throw an error.

\phpinputminted{02/figures/01/06-type-error}


\section{Polymorphism}

You might look at the above example and think, ``What's the point of passing in the \texttt{Mail} object?'' And you'd be right. If we can only pass in a \texttt{Mail} object, then we may as well just create it in the \texttt{MailingList} class. What if we wanted to be able to sometimes send our mailing list emails using the server's built-in mail program and sometimes using MailChimp? We can't type-hint two different classes, but if we don't type-hint at all then you could accidentally pass in something that will break your code completely.
\\

This is where the idea of \textbf{polymorphism} comes in. Polymorphism is when two \textit{different} types of object share enough in common that they can take each other's place in a specific context.
\\

There are two ways to achieve this in most OO languages:

\begin{itemize}
    \item \textbf{Inheritance}: when an object can inherit methods/properties from another object, creating a hierarchy of object types.
    \item \textbf{Interfaces}: when an object implements a defined set of methods.
\end{itemize}


\section{Inheritance}

Inheritance lets us create a hierarchy of object types, where the \textbf{children} types inherit all of the methods and properties of the \textbf{parent} classes. This allows reuse of methods and properties but allowing for different behaviours.
\\

For example, say that we want to create a \texttt{Mail} and a \texttt{MailChimp} class. Both of these will be responsible for sending an email, so they will have some things in common, but their inner workings will be different. We could create a parent \texttt{Mailer} class:

\phpinputminted{02/figures/01/07-Mailer}

We move all of the shared code into the \texttt{Mailer} class. We don't put the \texttt{send()} method into the \texttt{Mailer} class, as it will be different for each implementation. We can then \textbf{extend} the \texttt{Mailer} class to copy its behaviour into \texttt{Mail} and \texttt{MailChimp}:

\phpinputminted{02/figures/01/08-Mailer-children}

Now, in the \texttt{MailingList} class we can use \texttt{Mailer} as the type declaration. That means that depending on our mood,\footnote{Or possibly something more concrete} we can send messages with either the local mail server or with MailChimp:

\phpinputminted{02/figures/01/09-MailingListRedux}


\subsection{Abstract Classes}

But you can perhaps see a few issues with this. Firstly, you could create an instance of \texttt{Mailer} and pass that into \texttt{sendWith()}. This would cause an issue because the \texttt{Mailer} class doesn't have a \texttt{send()} method, so you'd get an error. Secondly, there's no guarantee that a child of \texttt{Mailer} has a \texttt{send()} method: we could easily create a child of \texttt{Mailer} but forget to add it. This would lead to the same issue:

\phpinputminted{02/figures/01/10-Mailer-oops}

This is where \textbf{abstract classes} come in. These are classes that are not meant to be used to create object instances directly, but instead are designed to be the parent of other classes. We can setup properties and methods in them, but they can't actually be constructed, only extended.
\\

We can also create \texttt{abstract} method signatures. These allow us to say that a child class \textit{has} to implement a method with the given name and parameters. We'll get an error if we don't.
\\

This gets rounds both issues: we won't be able to instance \texttt{Mailer} and we can make sure any of its children have a \texttt{send()} method:

\phpinputminted{02/figures/01/11-Mailer-redux}


\subsection{Overriding}

Children classes can \textbf{override} methods and properties from parent classes. That means if we wanted to make it so that the \texttt{to()} method in the \texttt{MailChimp} class did something slightly different, then we could write a different \texttt{to()} method. As long as it has \textit{the same method signature}\footnote{The \texttt{\_\_construct()} method is an exception to this rule. As it is unique to the specific class it can have a different set of parameters.} (i.e. excepts the same parameters), this will work:

\phpinputminted{02/figures/01/12-override}

If necessary it's possible to stop a child class from overriding a method by adding the \texttt{final} keyword in front of it:

\phpinputminted{02/figures/01/13-Mailer-final}

\subsection{\texttt{parent}}

Sometimes when we're overriding a method it can be useful to still have access to the parent object's version. For example, we might want to override the \texttt{to()} method of \texttt{Mailer}, but only to add a bit of functionality. We can do this by calling the method name on \texttt{parent}. Make sure you pass along the arguments when you do this:

\phpinputminted{02/figures/01/14-parent}

You can call the parent's constructor method with \texttt{parent::\_\_construct()}.


\subsection{The Intuitive Appeal of Inheritance}

You can probably see that there's an intuitive sort of appeal to inheritance. Programming books often give examples like creating an \texttt{Animal} class, which then has a child \texttt{Mammal} class, which then has a child \texttt{Primate} class, which has a child \texttt{Person} class, \&c.
\\

It also looks like we can save ourselves a lot of code by adding as many of the properties and methods as we can into the classes higher up the hierarchy: we only need to write the \texttt{to()} and \texttt{from()} methods \textit{once} in the \texttt{Mailer} class, which save a lot of repetition.





\section{Interfaces}

Interfaces are the other way that we can take advantage of polymorphism. An \textbf{interface} is a list of \textbf{method signatures} that a class can say it conforms to. It is a contract: if a class implements an interface then we are guaranteed that it has a certain set of methods taking a specified set of arguments.
\\

For example, rather than creating a \texttt{Mailer} abstract class, we could instead create an interface:

\phpinputminted{02/figures/01/15-MailerInterface}

You can see that we list out all of the methods that a class that implements this interface \textit{must} implement. We would use it as follows:

\phpinputminted{02/figures/01/16-Mail-interface}

We would use it in \texttt{MailingList} in exactly the same way:

\phpinputminted{02/figures/01/17-MailingList-interface}

Now we could only pass in classes that implement the \texttt{MailerInterface} interface.
\\

This might at first not seem like much of an advantage over using an abstract class, but a class can only \texttt{extend} a single class, whereas it can \texttt{implement} as many interfaces as it likes:

\phpinputminted{02/figures/01/18-MailChimp}


\subsection{Message Passing}

If you look at the \texttt{sendWith()} method you'll see that it only uses the \texttt{to()}, \texttt{from()}, and \texttt{send()} methods of the object that gets passed in. Therefore it's \textit{guaranteed} that if the object passed in conforms to the \texttt{MailerInterface}, which defines all three of those methods, then it will work. That's not to say the implementation will necessarily work, but the \texttt{sendWith()} method has access to all the methods that it requires.
\\

This is a core idea in OOP. But one that is often not talked about with beginner level books on OOP. Although it's called \textit{Object}-Oriented Programming, it's not actually the objects that should get the focus, it's the \textbf{messages} that they can send to one another: the methods and their parameters.
\\

The reason interfaces are such a powerful idea is because they focus solely on the messages and don't tell you anything about the implementation. This is important because of encapsulation. If we have to worry about how a specific object does something, then we can't treat it as a black box.
\\

This is also why try to use only \texttt{private} properties: if a property is \texttt{public} then we need to know about how the internals of the class work.
\\

Inheritance can break the idea of encapsulation because in order to use it you often need to know about the \textit{internal} workings of the parent classes: Were these properties declared? Has another class in the hierarchy overriden anything? This makes it much harder to work with and it gets worse as the codebase gets bigger. What if you need to change a parent class: will it break any of the children?
\\

``But,'' I hear you saying, ``you've got to write the same \texttt{to} and \texttt{from} methods and repeat all the properties in \texttt{Mail} and \texttt{MailChimp}! And repeating code is bad.''
\\

This is true with these very simple class examples, but you generally find that with more complex classes you don't actually repeat all that much code between them. If you do need to repeat code between classes there are often cleaner ways to do it than using inheritance: a shared class, for example.


\section{Inheritance Tax}

\quoteinline{The object-oriented version of ``Spaghetti code'' is, of course, ``Lasagna code'': too many layers}{Roberto Waltman}

If you read any books about OOP they'll focus a lot of their time on inheritance. While inheritance can be very useful, all of this attention means that it's often the technique that programmers reach for when they need to add a bit of functionality. And it will almost always lead to much more complicated code.
\\

That's not to say it isn't useful. It's totally fine to inherit code that frameworks/libraries provide, as in these cases you're generally adding one tiny bit of functionality to something that's much more complicated under-the-hood.\footnote{Although some purists would say even in this case there are better alternatives: see the \href{https://www.thoughtfulcode.com/orm-active-record-vs-data-mapper/}{Active Record vs Data Mapper} debate} But if it's code that you've written, it's always worthwhile thinking ``Do I really need to use inheritance?''
\\

Sandi Metz suggests not using inheritance until you have \textit{at least} three classes that are \textit{definitely} all using exactly the same methods. You should never start out by writing an abstract class: write the actual use-cases first and only write an abstract class if you definitely need one. If you do use inheritance then try not to create layers and layers of it: maybe have a rule that you'll only ever inherit through one layer.


\subsection{Composition}

\quoteinline{Prefer composition over inheritance}{The Gang of Four, \textit{Design Patterns}}

Think back to the \texttt{Animal}, \texttt{Mammal}, \texttt{Primate} \&c. example earlier. It intuitively has some appeal, but if you think about it for a while it starts to fall down. For example, if your mammal class had any functionality to do with giving birth to live young, then the \texttt{Monotreme} class is going to have issues. It turns out that despite all having the same common ancestor at some point in history, the only thing that \textit{all} mammals have in common is something to do with jaw articulation.\footnote{``Synapsids that possess a dentary–squamosal jaw articulation and occlusion between upper and lower molars with a transverse component to the movement'', Kemp, T. S. (2005) \textit{The Origin and Evolution of Mammals}}
\\

The idea of \textit{composition over inheritance} is that rather than sharing behaviour with inheritance, we share it using interfaces and shared classes. Rather than having a \texttt{giveBirthToLiveYoung()} method that all \texttt{Mammal} objects have to inherit, instead have a \texttt{GiveBirthToLiveYoung} interface, which is implemented only by those mammals that need to. If a lot of those classes share the same implementation of that method then consider moving it into a separate class that they can all share. It might require a bit more code to get working, but it is much easier to make changes to.

\begin{infobox}{The Law of Demeter}
    The ``Law of Demeter'' is a guideline for OOP about how objects should use other objects. Expressed succinctly:

    \begin{center}
        \textit{Each object should only talk to its friends; don't talk to strangers}
    \end{center}

    In practice, this means that an object should only call methods on either itself or objects that it has been given. You should avoid calling a method which returns an object and then calling a method on that object: it requires too much knowledge about other objects.

    \stdinputminted[firstline=3,frame=topline]{php}{02/figures/01/19-Demeter.php}
\end{infobox}


\section{Additional Resources}

\begin{itemize}[leftmargin=*]
    \item \href{https://phpapprentice.com/classes-inheritance.html}{PHP Apprentice: Inheritance}
    \item \href{https://phpapprentice.com/interfaces.html}{PHP Apprentice: Interfaces}
    \item \href{https://en.wikipedia.org/wiki/Composition\_over\_inheritance}{Wikipedia: Composition over Inheritance}
    \item \href{https://en.wikipedia.org/wiki/Law\_of\_Demeter}{Wikipedia: The Law of Demeter}
    \item \href{https://www.poodr.com}{Practical Object-Oriented Design in Ruby}: An intermediate book about OOP. In Ruby, but all the key concepts work in PHP too.
    \item \href{https://en.wikipedia.org/wiki/SOLID}{Wikipedia: The SOLID Principles of OOP}: Quite advanced, but incredibly useful if you can get your head round it.
\end{itemize}
